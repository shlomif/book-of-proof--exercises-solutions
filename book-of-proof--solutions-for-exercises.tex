\documentclass[a4paper]{article}

\usepackage[]{fullpage}

\usepackage[]{amsmath}

% This statement puts a one-line spacing between two adjacent paragraphs
\setlength\parskip{\medskipamount}

% This statement cancels the indentation of the paragraph's first line
\setlength\parindent{0pt}

\begin{document}

Like I said, it did not come to me right away and I had to think about it
for a while. I pondered various methods, and then came to think about the
question this way:

In each throw the numbers 1..6 can be subtracted. I first realized that I
could immediately subtract 1 from all $ 6^4 $ of the possible throws, because
one always subtract {\bf at least} 1. Furthermore, when is another 1 point
subtracted from the final sum? Obviously this is the case when 2 or more
points are subtracted. When does it happen? It happens when all the dice
are bigger than 1. (or else 1 would have been subtracted.) Since all the
dice are in the range 2-6 there are $5^4$ different possibilities for this
case.

The next point is subtracted when all the dice are in the range 3-6, hence
$4^4$ possibilities and so on. Now, to calculate the total, let's start from
the sum of all possible throws of 4 dice, without the minimal die removed.
This sum is equal to $6^4 \cdot 14 = 18144$. Then, let's remove $6^4 = 1296$
because of the first point removed, $5^4$ because of the second point etc.
Eventually we get:

$6^4 * 14 - 6^4 - 5^4 - 4^4 - 3^4 - 2^4 - 1^4 = 15869$

To find the average, all one has to do is divide it by $6^4$, which is the
number of individual throws. $\frac{15869}{6^4} = 12.24$, which is also the
number that my friend's program returned.

To generalise it for n dice each having the numbers 1..m on its side
(AD\&D) also involves using dice with 4, 8, 10,12, 20 and sometimes 100
sides. :-)) all one has to do is replace $6$ with $n$ and $4$ with $m$
where  appropriate. One should remember that the average throw for an
individual die of 1..m is $\dfrac{1+m}{m} $ and for n such dice
$\dfrac{n\left(1+m\right)}{m}$. We eventually get to:

$
\frac{\left( \frac{m^n \cdot n \cdot \left(1+m\right)}{m} \right) - m^n -
\left(m-1\right)^n - \left(m-2\right)^n - \left(m-3\right)^n ... - 1^n}{m ^ n}
$

Eliminating the next smallest die, and the other dice in order, is a bit
more tricky. As a matter of fact, it took me two more years to finally come
up with a solution. (not that I spent all my time thinking about it)

The basic idea is this: regarding the first point of the second least die,
it's obvious that we still have to remove $6^4$ from the total. About the
second point: it is removed only if all the dice, except maybe one has a
face value of 2 or more. Phrased in a different way it is removed:

\begin{enumerate}
\item If there are 4 dice in the range 2..6. {\bf Or}:
\item If there are 3 dice in the range 2..6 and one die whose value is 1.
\end{enumerate}

The conditions for the other 4 cases are similar: if the additional point
is the Kth than there could be either 4 dice in the range K..6 or 3 dice in
the range K..6 and one die in the range 1..(K-1). To evaluate it
mathematically, I'll use the standard formulas and eventually get to:

$\left(6-K+1\right)^4  +  \left( \left(6-K+1\right)^3 \cdot \left(K-1\right) \right) \cdot 4$


If we subtract $6^4$ and those 5 sums from the total sum that was acquired in
the previous stage, we'd get to the new total with the two least dice
removed on each throw. Divide it by the number of throws - $6^4$ - and we get
the new average. The grand formula, then, is:

$ \frac{
    6^4\cdot14-\sum\limits_{i=1}^{6}\left.i^4\right. -
    \sum\limits_{i=1}^{6}\left[\left(6-i+1\right)^4+\left(\left(6-i+1\right)^3
    \cdot \left(i-1\right)\right) \cdot 4 \right]
}{6^4}$

From the 3rd least dice onward, this method gets more and more complicated
because we have to consider 3 different cases for every point we
subtract, 4 cases for the fourth die, and so on. If anyone has an idea
on how to improve this method, or can suggest an alternative method
which is simpler as more and more dice are removed by order, please let
me know and I'll post it here.

I, for my part, am no longer interested in this problem, and do not deal
with it, at least not until I extend my knowledge in Combinatorics and the
Theory of Probability.

A final note: you can find a C program, not unlike the one my friend used,
that calculates the average of such throws
{\bf TODO Link me here}. It can do that for any number of
dice with any number of sides, and while eliminating any number of minimal
dice.
</p>

The program is quite stupid and merely iterates over all the
different throws, and adds the sum of every throw to the total. It
becomes less efficient as the number of sides and/or the number of dice
increases.

Writing a program that implements my method {\bf for any number of removed
dice} is a bit complicated because the formula itself gets more and more
complicated. Maybe I'll get to it one day.

\end{document}

