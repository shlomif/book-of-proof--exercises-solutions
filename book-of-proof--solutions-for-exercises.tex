\documentclass[a4paper]{article}

\usepackage[]{fullpage}

\usepackage[]{amsmath}
\usepackage[]{amssymb}

% This statement puts a one-line spacing between two adjacent paragraphs
\setlength\parskip{\medskipamount}

% This statement cancels the indentation of the paragraph's first line
\setlength\parindent{0pt}

\begin{document}

\section{Chapter 4}

\subsection{Exercise 1}

Given: x is even.

x is even so $x = 2a$ for some $ a \in Z $ . Therefore,
$ x^2 = (2a)^2 = 4a^2 = 2\cdot(2\cdot a^2)$. As a result, $x^2 = 2b$
for some integer $b = 2 \cdot a^2 $ and is even.

\subsection{Exercise 2}

Given: x is an odd integer.

Since x is odd, then it can be written as $x = 2a+1$ for some $a \in Z$.

As a result:
$x^3 = (2a+1)^3 = 8a^3 + 12a^2 + 6a + 1 = 2\left(4a^3+6a^2+3a\right) + 1 $.

One can see that $x^3 = 2b+1$ for $b = 4a^3+6a^2+3a$ and is therefore odd.

Q.E.D.

\subsection{Exercise 3}

Given: $a$ is an odd integer.

Since $a$ is odd, it can be written as $a = 2x+1$ for some $x \in \mathbb{Z}$.

As a result: $a^2 + 3a + 5 = (2x+1)^2 + 3(2x+1) + 5 = 4x^2+10x+9 =
2\left(2x^2+5x+4\right) + 1$ .

Therefore the expression is of the form $2y+1$ where
$y = \left(2x^2+5x+4\right)$ and $y \in \mathbb{Z}$ and it is odd.

\subsection{Exercise 4}

Given: x is odd, and y is odd.

Since x and y are odd then by definition they can be expressed as $ x = 2a+1 $
and $ y = 2b+1$ for some $ a, b \in \mathbb{Z} $.

As a result $xy = (2a+1)(2b+1) = 2ab + 2a + 2b + 1 = 2(ab+a+b)+1$.

So $xy$ can be written as $2c+1$ for some $c=(ab+a+b)$ and $c \in \mathbb{Z}$
and is therefore odd.

\subsection{Exercise 5}

Given: $ x, y \in \mathbb{Z} $ , and $x$ is even.

Since $x$ is even it can be written as $2a$ for some $a \in \mathbb{Z}$.

As a result $xy = 2ay = 2(ay)$.

One can see that $xy$ can be written as $2c$ where $c = ay$ and is therefore
even.

\subsection{Exercise 6}

Since $a | b$ then $b = xa$ for some $x \in \mathbb{Z}$ , and similarly $a | c$
implies $c = ya$.

As a result: $b+c = xa + ya = (x+y)a$ where $ (x+y) \in \mathbb{Z}$ and
therefore $a|(b+c)$.

\subsection{Exercise 7}

Given: $a | b$.

Since $a | b$ then by definition, $b = xa$.

As a result $b^2 = (xa)^2 = (x^2)(a^2)$.

and as a result, $b^2 = y(a^2)$ where $y=(x^2)$ and $a^2|b^2$.

\end{document}

